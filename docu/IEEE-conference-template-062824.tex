\documentclass[conference]{IEEEtran}
\IEEEoverridecommandlockouts

\usepackage{cite}
\usepackage{amsmath,amssymb,amsfonts}
\usepackage{algorithmic}
\usepackage{graphicx}
\usepackage{textcomp}
\usepackage{xcolor}
\usepackage[croatian]{babel}


\def\BibTeX{{\rm B\kern-.05em{\sc i\kern-.025em b}\kern-.08em
    T\kern-.1667em\lower.7ex\hbox{E}\kern-.125emX}}
\begin{document}

\title{Generiranje imena naselja pomocu LSTM mreže}


\author{
\IEEEauthorblockN{Antonio Čogelja}
\and
\IEEEauthorblockN{Morena Granić}
\and
\IEEEauthorblockN{Fran Lubina}
\and
\IEEEauthorblockN{Iva Jurković}
\and
\IEEEauthorblockN{Jakov Juvančić}
\and
\IEEEauthorblockN{Matej Logarušić}
}

\maketitle

\begin{abstract}
Cilj projekta je LSTM rekurzivna neuronska mreža na razini znakova koja generira realistična imena hrvatskih naselja. Fokus projekta je treniranje i razvijanje neuronske mreže za generiranje realističnih imena hrvatskih naselja. Korištenjem LSTM mreže, koja je prilagođena za analizu sekvencijskih podataka, cilj je razviti model sposoban za učenje jezičnih obrazaca i struktura iz postojećih imena naselja. Svrha mreže je generiranje novih imena temeljenih na tim naučenim obrascima, pri čemu se zadržavaju jezične i strukturne zakonitosti specifične za taj kontekst.
Željena točnost modela $\eta$ je $\lim_{\tau \to 0} \eta = 0.5$
\end{abstract}

\begin{IEEEkeywords}
Naselje, LSTM, rekurzivne mreže, neuronske mreže-
\end{IEEEkeywords}

\section{Uvod}
Ishod projekta je LSTM rekurzivna neuronska mreža na razini znakova koja generira realistična imena hrvatskih naselja.\\
Mreža radi sa vektorima koji predstavljaju slova hrvatske abecede proširene specijalnim znakovima $\Sigma = \{ \text{hrv. abeceda}\} \cup \{\langle start \rangle, "\setminus 0"\}$.\\
Ulaz mreže je one-hot vektor $\mathbf{x}^{(t)}$ dimezije $\lvert \Sigma \rvert = 30 + 2$.
\begin{equation}
\mathbf{x}^{(t)}_i=
    \begin{cases}
      1, & \text{ako}\ i=j \\
      0, & \text{inače}
    \end{cases}
\end{equation}
Izlaz dobiven na kraju pojedinog vremenskog koraka $t$ je vektor vjerojatnosti pojave pojednog znaka abecende.\\
\begin{align}
    \hat{\mathbf{y}}^{(t)} &= \begin{bmatrix}
           p(c_0) \\
           p(c_1 | c_0) \\
           \vdots \\
           p(c_{\lvert \Sigma \rvert -1} | \bigcap_{i=0}^{\lvert \Sigma \rvert -2} c_i)
         \end{bmatrix}
         \quad \quad \text{Gdje} \quad c \in \Sigma
\end{align}
Vjerojatnosti su dobivene softmax funkcijom parametriziranom hiperparametrom temperature $\tau$.\\
Na temelju tih vjerojatnosti se uzorkuje konačni izlazni vektor $\mathbf{y}^{(t)}$, odnosno t-ti znak u imenu naselja.\\
\begin{equation}
 \mathbf{y}^{(t)} \sim \hat{\mathbf{y}}^{(t)} = \sigma_{\tau}(f(\mathbf{x}^{(t)} ; \boldsymbol{\theta}))
\end{equation}
\ \\
$f(\mathbf{x} ; \boldsymbol{\theta})$ predstavlja ukupno djelovanje ćelija modela nad njenim ulazom parametrizirano hiperparametrima modela $\boldsymbol{\theta} = \begin{bmatrix} \lvert \mathbf{a} \rvert & \mu & \tau \end{bmatrix}$\\ (opisani u poglavlju \ref{subsect_hiper})
Temperaturno uzorkovanje je izabrano, jer omogućava eksperimentiranje i generiranje zanimljivih toponima.
Izlaz mreže je niz znakova $\{\mathbf{y}^{(t)}\} \biggr \rvert_{t=0}^{T-1}$, odnosno ime naselja.
Željena točnost modela $\eta$ je $\lim_{\tau \to 0} \eta = 0.5$

\section{Pregled literature}
--

\section{Opis implementirane LSTM mreže}
Fokus projekta je treniranje i razvijanje neuronske mreže za generiranje realističnih imena hrvatskih naselja. Korištenjem LSTM mreže, koja je prilagođena za analizu sekvencijskih podataka, cilj je razviti model sposoban za učenje jezičnih obrazaca i struktura iz postojećih imena naselja. Svrha mreže je generiranje novih imena temeljenih na tim naučenim obrascima, pri čemu se zadržavaju jezične i strukturne zakonitosti specifične za taj kontekst.
LSTM ćelija i mreža je implementirana u radnom okviru pyTorch.\\
Dizajn mreže i podešavanje hiperparametara se odvija paralelno sa implementacijom mreže u radnom okviru Keras.

\subsection{Arhitektura}
--

\subsection{Hiperparametri}
\label{subsect_hiper}
\begin{enumerate}
\item Dimenzija skrivenog stanja: $\lvert \mathbf{a} \rvert$
\item Stopa učenja: $\mu$
\item Temperatura: $\tau$
\item Broj LSTM ćelija
\end{enumerate}

\subsection{Ćelija}
Ćelija izgledda ovako.

\subsection{Treniranje}
BPTT je korišten kao algoritam učenja.
Kao funkcija gubitka koristi se unakrsna entropija.
\begin{equation}
L_i = - \sum_{t = 0}^{T-1} \mathbf{x}_i^{(t)} \cdot log(\hat{\mathbf{y}}_i^{(t)})
\end{equation}
inačica BPTT koju mi koristimo je u biti propagirani stohastički gradijentni spust.

\section{Opis eksperimentalnih rezultata}\label{FAT}
\paragraph{optimiranje hiperparametara}
--

\begin{table}[htbp]
\caption{Table Type Styles}
\begin{center}
\begin{tabular}{|c|c|c|c|}
\hline
\textbf{Table}&\multicolumn{3}{|c|}{\textbf{Table Column Head}} \\
\cline{2-4} 
\textbf{Head} & \textbf{\textit{Table column subhead}}& \textbf{\textit{Subhead}}& \textbf{\textit{Subhead}} \\
\hline
copy& More table copy$^{\mathrm{a}}$& &  \\
\hline
\multicolumn{4}{l}{$^{\mathrm{a}}$Sample of a Table footnote.}
\end{tabular}
\label{tab1}
\end{center}
\end{table}

\subsection{Usporedba rezultata}
--

\section{Zaključak}
--


\begin{thebibliography}{00}
\bibitem{b1} G. Eason, B. Noble, and I. N. Sneddon, ``On certain integrals of Lipschitz-Hankel type involving products of Bessel functions,'' Phil. Trans. Roy. Soc. London, vol. A247, pp. 529--551, April 1955.
\end{thebibliography}



\end{document}
