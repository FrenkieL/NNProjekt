\documentclass{report}
\usepackage{enumitem}
\usepackage[table]{xcolor}
\usepackage[croatian]{babel}
\usepackage{longtable}
\usepackage{tikz}
\usepackage[most]{tcolorbox}
\usepackage{lscape}
\usepackage{float}
\usepackage{multirow}

\renewcommand{\thesection}{(\Roman{section}.) }

\title{Generiranje naselja pomoću LSTM mreže}

\author{Čogelja, Granić, Lubina, Jurković, Juvančić, Logarušić}

\begin{document}

\maketitle

\chapter*{prijedlog}
\ \\
\\
\\
\\

\section{Ishod projekta}
Ishod projekta je LSTM rekurzivna neuronska mreža na razini znakova koja generira realistična imena hrvatskih naselja.\\
\\
Mreža radi sa vektorima koji predstavljaju slova hrvatske abecede proširene specijalnim znakovima $\Sigma = \{ \text{hrv. abeceda}\} \cup \{\langle start \rangle, "\setminus 0"\}$.\\
Ulaz mreže je one-hot vektor $\mathbf{x}^{(t)}$ dimezije $\lvert \Sigma \rvert = 32$.
\begin{equation}
\mathbf{x}^{(t)}_i=
    \begin{cases}
      1, & \text{ako}\ i=j \\
      0, & \text{inače}
    \end{cases}
\end{equation}
Izlaz dobiven na kraju pojedinog vremenskog koraka $t$ je vektor vjerojatnosti pojave pojednog znaka abecende.\\
\begin{align}
    \hat{\mathbf{y}}^{(t)} &= \begin{bmatrix}
           p(c_0) \\
           p(c_1 | c_0) \\
           \vdots \\
           p(c_{\lvert \Sigma \rvert -1} | \bigcap_{i=0}^{\lvert \Sigma \rvert -2} c_i)
         \end{bmatrix}
         \quad \quad \text{Gdje} \quad c \in \Sigma
\end{align}
Vjerojatnosti su dobivene softmax funkcijom parametriziranom hiperparametrom temperature $\tau$.\\
Na temelju tih vjerojatnosti se uzorkuje konačni izlazni vektor $\mathbf{y}^{(t)}$, odnosno t-ti znak u imenu naselja.\\
\begin{equation}
 \mathbf{y}^{(t)} \sim \sigma_{\tau}(\hat{\mathbf{y}}^{(t)})
\end{equation}

Temperaturno uzorkovanje je izabrano, jer omogućava eksperimentiranje i generiranje zanimljivih toponima.\\
\\
Izlaz mreže je niz znakova $\{\mathbf{y}^{(t)}\} \biggr \rvert_{t=0}^{T-1}$, odnosno ime naselja.

\section{Tema i kratki opis}
Tema projekta je generiranje realističnih imena hrvatskih naselja.\\
U tu svrhu će se izraditi LSTM mreža parametrizirana sljedećim hiperparametrima:
\begin{enumerate}
\item Dimenzije skrivenog stanja: $\lvert \mathbf{a} \rvert$
\item Stopa učenja: $\mu$
\item Temperatura: $\tau$
\item Broj LSTM jedinica
\end{enumerate}

\section{Zadatci na projektu i raspodjela posla}
Ostvarenje projekta podrazumijeva slijedeće zadatke:
\begin{longtable}{|p{130pt}| p{110pt} |p{200pt}|}
\hline
 & \textbf{ime} & \textbf{opis}\\
\hline
\multirow{8}{*}{pisani rad} & kratki uvod & N/A\\ \cline{2-3}
 & opis problema & N/A \\ \cline{2-3}
 & opis eksperimentalnih rezultata & N/A \\ \cline{2-3}
 & diskusija i usporedba rezultata & N/A \\ \cline{2-3}
 & lektoriranje & N/A \\ \cline{2-3}
 & zaključak & N/A \\ \hline
\multirow{1}{*}{administrativni poslovi} & & Lubina\\ \hline
\multirow{1}{*}{izrada prezentacije} &  & N/A\\ \hline
\multirow{5}{*}{implementacija} & priprema skupa podataka & N/A  \\ \cline{2-3} 
 & implementacija mreže & N/A \\ \cline{2-3}
 & implementacija uzorkovanja & N/A \\ \hline
\multirow{2}{*}{treniranje} & implementacija algoritma učenja & N/A  \\ \cline{2-3} 
 & treniranje & N/A \\ \hline
\multirow{3}{*}{validacija} & ručna validacija modela & N/A  \\ \cline{2-3} 
 & podešavanje hiperparametara & N/A  \\ \hline 
 
\caption{zadatci i raspored}

\end{longtable}
\ \\
Mreža će biti implementiarana pomoću TensorFlow radnog okvira.\\
Podešavanje hiperparametara se odvija paralelno sa implementacijom mreže u radnom okviru Keras.\\



\end{document}

