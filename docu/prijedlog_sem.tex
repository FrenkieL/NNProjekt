\documentclass{report}
\usepackage{enumitem}
\usepackage[table]{xcolor}
\usepackage[croatian]{babel}
\usepackage{longtable}
\usepackage{tikz}
\usepackage[most]{tcolorbox}
\usepackage{lscape}
\usepackage{float}
\usepackage{multirow}

\renewcommand{\thesection}{(\Roman{section}.) }

\title{Generiranje imena naselja pomoću LSTM mreže}

\author{Čogelja, Granić, Lubina, Jurković, Juvančić, Logarušić}

\begin{document}

\maketitle

\chapter*{prijedlog}
\ \\
\\
\\
\\

\section{Ishod projekta}
Ishod projekta je LSTM rekurzivna neuronska mreža na razini znakova koja generira realistična imena hrvatskih naselja.\\
\\
Mreža radi sa vektorima koji predstavljaju slova hrvatske abecede proširene specijalnim znakovima $\Sigma = \{ \text{hrv. abeceda}\} \cup \{\langle start \rangle, "\setminus 0"\}$.\\
Ulaz mreže je one-hot vektor $\mathbf{x}^{(t)}$ dimezije $\lvert \Sigma \rvert = 32 + 2$.
\begin{equation}
\mathbf{x}^{(t)}_i=
    \begin{cases}
      1, & \text{ako}\ i=j \\
      0, & \text{inače}
    \end{cases}
\end{equation}
Izlaz dobiven na kraju pojedinog vremenskog koraka $t$ je vektor vjerojatnosti pojave pojednog znaka abecende.\\
\begin{align}
    \hat{\mathbf{y}}^{(t)} &= \begin{bmatrix}
           p(c_0) \\
           p(c_1 | c_0) \\
           \vdots \\
           p(c_{\lvert \Sigma \rvert -1} | \bigcap_{i=0}^{\lvert \Sigma \rvert -2} c_i)
         \end{bmatrix}
         \quad \quad \text{Gdje} \quad c \in \Sigma
\end{align}
Vjerojatnosti su dobivene softmax funkcijom parametriziranom hiperparametrom temperature $\tau$.\\
Na temelju tih vjerojatnosti se uzorkuje konačni izlazni vektor $\mathbf{y}^{(t)}$, odnosno t-ti znak u imenu naselja.\\
\begin{equation}
 \mathbf{y}^{(t)} \sim \hat{\mathbf{y}}^{(t)} = \sigma_{\tau}(f(\mathbf{x}^{(t)} ; \boldsymbol{\theta}))
\end{equation}
\ \\
$f(\mathbf{x} ; \boldsymbol{\theta})$ predstavlja ukupno djelovanje ćelija modela nad njenim ulazom parametrizirano hiperparametrima modela $\boldsymbol{\theta} = \begin{bmatrix} \lvert \mathbf{a} \rvert & \mu & \tau \end{bmatrix}$\\ (opisani u poglavlju \ref{sec_tema})
\ \\
Temperaturno uzorkovanje je izabrano, jer omogućava eksperimentiranje i generiranje zanimljivih toponima.\\
\\
Izlaz mreže je niz znakova $\{\mathbf{y}^{(t)}\} \biggr \rvert_{t=0}^{T-1}$, odnosno ime naselja.\\
\\
Željena točnost modela $\eta$ je kada $\lim_{\tau \to 0} \eta = 0.5$

\section{Tema i kratki opis}
\label{sec_tema}
Fokus projekta je treniranje i razvijanje neuronske mreže za generiranje realističnih imena hrvatskih naselja. Korištenjem LSTM mreže, koja je prilagođena za analizu sekvencijskih podataka, cilj je razviti model sposoban za učenje jezičnih obrazaca i struktura iz postojećih imena naselja. Svrha mreže je generiranje novih imena temeljenih na tim naučenim obrascima, pri čemu se zadržavaju jezične i strukturne zakonitosti specifične za taj kontekst. U planu je karakterizirati mrežu sa sljedećim hiperparametrima:
\begin{enumerate}
\item Dimenzije skrivenog stanja: $\lvert \mathbf{a} \rvert$
\item Stopa učenja: $\mu$
\item Temperatura: $\tau$
\item Broj LSTM ćelija
\end{enumerate}
\ \\
LSTM ćelija i mreža će biti implementirane u radnom okviru pyTorch.\\
Dizajn mreže i podešavanje hiperparametara se odvija paralelno sa implementacijom mreže u radnom okviru Keras.\\
Točan izgled ćelije i dizajn mreže će biti određeni naknadno.\\
\\
BPTT će biti korišten kao algoritam učenja.\\
Funkcija pogreške će biti određena naknadno.\\

\section{Zadatci na projektu i raspodjela posla}
Ostvarenje projekta podrazumijeva slijedeće zadatke:
\begin{longtable}{|p{100pt}| p{110pt} |p{20pt}| p{80pt}|}
\hline
 & \textbf{Zadatak} & \textbf{ETA} & \textbf{Developeri}\\
\hline
\multirow{8}{*}{Dokumentacija} & Uvod & 5h \\ \cline{3-3}
 & Opis problema & 3h \\ \cline{2-3}
 & Opis eksperimentalnih rezultata & 1d & Grupa \\ \cline{2-3}
 & Diskusija i usporedba rezultata & 1d \\ \cline{2-3}
 & Lektoriranje & 1d \\ \cline{2-3}
 & Zaključak & 4h \\ \hline
\multirow{2}{*}{Administrativni poslovi} & Održavanje GitHub-a & & Lubina, Jurković\\ \cline{2-3} & Sastanci & & Grupa \\ \hline
\multirow{1}{*}{Izrada prezentacije} &  & 3d & Jurković \\ \hline
\multirow{3}{*}{Implementacija} & Obrada ulaznog skupa podataka & 1w & Granić,  Logarušić, Lubina \\ \cline{2-3}
& Implementacija modela & 1w & Jurković, Čogelja, Logarušić \\ \hline
& Dizajn modela u kerasu & 1w & Lubina, Juvančić, Granić \\ \cline{2-3}
\multirow{2}{*}{Treniranje}
 & Treniranje modela & 2d & Jurković, Čogelja, Logarušić\\ \hline
\multirow{3}{*}{Validacija} & Ručna validacija modela & 1w & Jurković, Granić,  Logarušić\\ \cline{2-3} 
 & Podešavanje hiperparametara u kerasu & 1w & Lubina, Juvančić, Granić  \\ \hline 
 
\caption{Zadaci i estimacije}

\end{longtable}
\section{Vremenski plan rada}

U priloženoj tablici prikazan je okviran plan rada koji obuhvaća ključne datume i zadatke koji su planirani u sklopu projekta. Rokovi su estimirani uzimajući u obzir praznike i ispitne rokove (MI i ZI) kako bi se mogao predvidjeti realan tok aktivnosti. Rokovi su isto tako fleksibilni zbog akademskih obaveza članova grupe što osigurava aktivno sudjelovanje svih članova. Mogući iterativni postupci promjene dizajna mreže i/ili ćelije te ispravljanje grešaka nisu bili mogući za opisati, ali su uzeti u obzir kao i, ispravljanje raznih grešaka.

\begin{longtable}{|p{35pt}|p{90pt}|p{90pt}|p{90pt}|p{90pt}|}
\hline
\textbf{Rok} & \textbf{30.10.} & \textbf{15.12.} & \textbf{6.1.} & \textbf{15.1.} \\
\hline
\multirow{5}{*}{\textbf{Zadaci}} 
                   & 1. Početak rada & 1. Obrada ulaznog skupa podataka & 1. Validacija modela & 1. Pisanje dokumentacije \\
                   & & 2. Dizajn modela u kerasu & 2. Podešavanje hiperparametara u kerasu & 2. Priprema prezentacije \\
                   & & 3. Implementacija modela u TensorFlow-u & & \\
                   & & 4. Treniranje modela & & \\
\hline

\caption{Planirani tok rada na projektu}

\end{longtable}



\end{document}

